\documentclass{book}
\usepackage[portuguese]{babel}
\usepackage{listings, xcolor, unitsdef}  % ! Need to fix the unities

\definecolor{backgroundcodigo}{RGB}{240,240,240}

\lstdefinestyle{ex1linha}{
  backgroundcolor=\color{backgroundcodigo},
  basicstyle=\ttfamily,%\footnotesize,
  % numbersize=\footnotesize,
  numbers=left,
  numbersep=10pt,
  tabsize=2,
  xleftmargin=5pt,
  framexleftmargin=5pt
}

\begin{document}

\tableofcontents

\chapter{O banco de dados}
\section{Informações pessoais}
\subsection{Nome}
Os nomes são gerados aleatoriamente de acordo com listas de frequência, sendo que o primeiro nome depende do sexo do indivíduo. As listas foram retiradas de...

\subsection{Idade}
A idade é representada por um número inteiro de anos, variando de 0 a 99. Para obter idades mais precisas (como meses, úteis para bebês), pode-se usar a data de aniversário.

\subsection{Data de nascimento}
A data de nascimento é representada no formato dd/mm/aaaa. No momento, não existe nascimento no dia 29 de fevereiro.

\subsection{Signo do zodíaco}
O signo do zodíaco é representado por uma palavra, dependente da data de nascimento.

\subsection{Signo chinês}
O signo chinês é representado por uma palavra, dependente do ano de nascimento. Embora o signo chinês correto não depende apenas do ano, mas também do dia do nascimento, esse detalhe não está presente nesse banco de dados.

\subsection{Gênero}
O gênero é representado por uma letra: ``M'' para masculino e ``F'' para feminino.

\subsection{Altura}
A altura é representada em centímetros, com precisão de 1 casa decimal. Para obter a altura em metros, basta dividir a coluna (ou a célula) por 100.

\subsection{Peso}
O peso é representado em kg, com precisão de 2 casas decimais.

\subsection{IMC}
O IMC é representado em kg/m\textsuperscript{2}, com precisão de 2 casas decimais.

\subsection{Percentual de gordura corporal (BF\%)}
O percentual de gordura corporal é representado em porcentagem, com 1 casa decimal.

\subsection{Raça}
A raça é representada por uma palavra. São consideradas 5 raças:
\begin{itemize}
  \item Branco
  \item Preto
  \item Amarelo
  \item Pardo
  \item Indígena
\end{itemize}
% ! Talvez "cor" não seja o nome mais adequado, talvez o melhor seja "raça"

\subsection{Cor do cabelo}
A cor do cabelo é representada por uma ou mais palavras. São consideradas 9 cores:
\begin{itemize}
  \item Loiro comum  % ? Eu deveria mesmo deixar o "comum"?
  \item Castanho claro
  \item Castanho médio
  \item Castanho escuro
  \item Castanho-ruivo
  \item Ruivo
  \item Preto
  \item Grisalho
  \item Branco
\end{itemize}

\subsection{Cor dos olhos}
A cor dos olhos é representada por uma palavra. São consideradas 4 cores:
\begin{itemize}
  \item Castanho
  \item Azul
  \item Âmbar
  \item Verde
\end{itemize}

\subsection{Cidade e estado}
Cidade e estados são representados em duas colunas, por palavras. Cuidado, pois existem cidades de mesmo nome em mais de um estado. Para selecionar uma cidade específica, filtre por cidade e estado, ou crie uma coluna que agregue as duas informações. Uma sugestão é criar uma coluna no modelo ``Cidade (UF)''.

\subsection{Estado civil}
Existem 4 estados civis: solteiro, casado, divorciado e viúvo. Indivíduos com menos de 16 anos não podem se casar legalmente, então recebem NA. Em algumas abordagens, pode ser interessante substituir NA por ``Solteiro(a)''.

\section{Personalidade e preferências}

\subsection{Personalidade}
A personalidade é descrita por 3 palavras, separadas por vírgula e espaço. A maneira mais fácil de fazer análises nessa coluna é separar os 3 traços.

\subsection{Time de futebol}
O time de futebol preferido é dado por um termo. São considerados 26 times, outros times, ou não torcer.

\subsection{Destino de viagem}
Existem 3 tipos de destino de viagem preferidos: praia, campo e cidade.

\subsection{Período do dia}
Existem 3 períodos do dia preferidos: manhã, tarde e noite.

\subsection{Refeição}
Existem 3 refeições preferidas: café da manhã, almoço e jantar.

\subsection{Prato}
Existem vários pratos preferidos. Indivíduos com NA não têm prato preferido.

\subsection{Visão de mundo}
Existem 3 visões de mundo consideradas: negativa, realista e positiva.

\subsection{Cor preferida}
Existem 11 cores favoritas, todas de uma só palavra.

\section{Dados socioeconômicos}

\subsection{Alfabetização}
A alfabetização é representada por uma letra: ``S'' representa indivíduos alfabetizados, e ``N'' representa indivíduos analfabetos.

\subsection{Escolaridade}
A escolaridade, ou nível de instrução, é representada por um termo ou por NA. Indivíduos com NA têm menos de 6 anos, e por isso não são considerados ``Sem instrução''. Dependendo da análise, recomenda-se substituir NA por ``Sem instrução''. Os níveis de escolaridade são:
\begin{itemize}
  \item Sem instrução
  \item Ensino fundamental incompleto
  \item Ensino fundamental completo
  \item Ensino médio incompleto
  \item Ensino médio completo
  \item Ensino superior incompleto
  \item Ensino superior completo
\end{itemize}

\subsection{Acesso à educação privada}
O acesso à educação privada se refere a escolas públicas ou privadas. É representado por uma palavra; indivíduos que não estão em idade escolar recebem NA.

\subsection{Emprego}
% ! ADICIONAR PARTE SOBRE A CHAVE DE TRANSFORMAÇÃO
O emprego é representado por um código numérico no \textbf{formato de palavra}. Esse código numérico é advindo da Classificação Brasileira de Ocupações (CBO), do Ministério do Trabalho. Uma chave com todos os códigos e seus nomes principais (pois alguns códigos são usados para um grupo de ocupações semelhantes) está disponível. ``Do lar'' representa indivíduos sem ocupação oficial, mas que cuidam da própria casa. Indivíduos desempregados, que não se ocupam com a própria residência são representados por NA.

\subsection{Renda}
A renda está representada por um número no formato R\$reais,centavos. Indivíduos que não têm renda recebem NA.

\subsection{Velocidade de acesso à internet}
A velocidade da internet à qual o indivíduo tem acesso é representada por em Mbps, com precisão de 1 casa decimal. Indivíduos sem acesso à internet recebem NA.

\subsection{Plano de saúde privado}
O acesso a um plano de saúde privado é representado por uma letra: ``S'' representa indivíduos com um plano de saúde privado, e ``N'' representa indivíduos sem acesso a plano de saúde privado.

\section{Saúde}

\subsection{Atividade esportiva}
A prática de atividade esportiva é representada em duas colunas. A primeira indica o esporte práticado pelo indivíduo, sendo que indivíduos que não praticam qualquer esporte recebem o valor "Sedentário". A segunda indica o tempo médio em horas, por semana, de prática do esporte (com precisão de uma casa decimal); indivíduos sedentários recebem o valor NA.

\subsection{Tipo sanguíneo}
O tipo sanguíneo é representado por um termo: o primeiro se refere ao tipo no sistema ABO, e o segundo no sistema Rh. Dependendo da análise desejada, pode ser conveniente separar os dois sistemas em colunas diferentes.

\section{Hemograma}

\subsection{Hemácias (Hc)}
Contagem de hemácias, em milhões por microlitro (\per{mi}{\microliter}).

\subsection{Hemoglobina (Hb)}
Quantidade de hemoglobina, em gramas por decilitro (\per{\gram}{\deciliter}).

\subsection{Hematórcrito (Ht)}
Proporção de hemácias no sangue, em porcentagem (\percent).

\subsection{Volume Corpuscular Médio (VCM)}
Volume médio das hemácias, calculado por \(\frac{Ht \times 10}{Hc}\), em fentolitros (\femtoliter).

\subsection{Leucócitos}
O número total de leucócitos em unidades por microlitro (\per{}{\microliter}).

\subsubsection{Neutrófilos, Eosinófilos, Basófilos, Linfócitos e Monócitos}
Tanto em número total, em unidades por microlitro (\per{}{\microliter}), quanto em proporção de leucócitos, em percentual (\percent).

\chapter{Técnicas úteis}
\section{Tratar NAs}
Algumas colunas do banco de dados têm "dados faltantes". No R, dados faltantes são representados por ``NA'' (Not Available). Em uma análise de dados, é importante definir como lidar com dados indisponíveis, já que nem sempre se pode simplesmente ignorar os dados indisponíveis. No mínimo, em alguns casos, eles podem diminuir o tamanho amostral de parte da análise; em outros, podem afetar a validade da análise se não forem considerados.
\par
Muitas funções não ignoram NAs automaticamente, como uma forma de forçar o usuário a considerar como tratá-los, além de deixar evidente caso um conjunto de dados tenha algum NA perdido no meio, que poderia passar despercebido. Um exemplo é a função \texttt{sum()}. Por padrão, essa função retorna NA se encontrar qualquer NA dentro dos dados entregues a ela. Para contornar isso, basta passar o argumento \texttt{na.rm = TRUE}. Muitas funções têm um parâmetro semelhante.
\par
A permanência de NAs na sua análise depende da sua seleção de dados. Por exemplo: se você decidir analisar pessoas com renda maior que 1 salário mínimo mensal, naturalmente vai retirar os NAs da coluna. Porém, se você fizer uma amostragem de mulheres pretas da cidade de São Paulo, e decidir analisar as rendas, provavelmente vai encontrar alguns NAs no meio dos seus dados. Dependendo da sua análise, considere de que forma a falta de alguns dados afeta a interpretação. De forma alguma ignore os NAs sem ter certeza que faz sentido ignorá-los.

\subsection{Códigos para lidar com NAs}
\subsubsection{Verificando a existência de NAs}
Você pode procurar se existe ao menos um NA em um conjunto de dados com o operador \texttt{\%in\%}, o que retorna TRUE ou FALSE:

\lstset{style=ex1linha}
\begin{lstlisting}[language=R]
NA %in% data
\end{lstlisting}

Você também pode encontrar quais valores são NA com \texttt{is.na()} (retorna TRUE ou FALSE, ou uma sequência deles). Essa função pode servir de filtro, também (no primeiro caso, mostra apenas os dados que são NA; no segundo, apenas os que não são):

\lstset{style=ex1linha}
\begin{lstlisting}[language=R]
is.na(data)
data[is.na(data)]
data[!is.na(data)]
\end{lstlisting}

\subsubsection{Lidando com NAs nativamente}
Ao invés da opção \texttt{na.rm}, algumas funções usam \texttt{na.action}. Nesses casos, elas utilizam a ação padrão atribuída nas opções do R, mas permitem uma mudança local nessas opções (como a função \texttt{lm()}):

\begin{itemize}
  \item[\textbf{\texttt{na.omit/na.exclude}}] Remove as observações que contém NA e retorna o objeto
  \item[\textbf{\texttt{na.pass}}] Retorna o objeto sem modificar
  \item[\textbf{\texttt{na.fail}}] Retorna o objeto apenas se não tiver NA
\end{itemize}

\subsubsection{Lidando com NAs usando o Tidyr (do Tidyverse)}
O pacote Tidyr oferece 3 formas de lidar com NAs:

\begin{itemize}
  \item[\textbf{\texttt{drop\_na()}}] Remove linhas contenho NAs nas colunas apontadas
  \item[\textbf{\texttt{fill()}}] Completa com o valor que veio antes ou que vem depois (dependendo da escolha)
  \item[\textbf{\texttt{replace\_na()}}] Substitui os NAs pelo valor escolhido.
\end{itemize}

Note que você pode utilizar \texttt{replace\_na()} para colocar um valor calculado, como a mediana ou a média dos valores disponíveis.

\section{Manipulação de datas}
\section{Manipulação de texto}
\section{Criação de colunas}  % Separando um dado em mais de uma coluna

\end{document}